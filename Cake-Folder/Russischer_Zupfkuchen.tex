\input{header.tex}

\begin{document}

\titel[für eine 26er Springform]{Russischer Zupfkuchen}
\begin{rezept}
	
	\schritt{Für den Teig}
	\zutat{300 g Mehl}{mit}
	\zutat{30 g Kakaopulver, ungesüßt}{und}
	\zutat{2 gestr. TL Backpulver }{in einer Schüssel mischen.}
	\zutat{150 g Zucker}{und}
	\zutat{1 Pck. Vanillezucker}{und}
	\zutat{1 Ei}{und}
	\zutat{150 g Butter}{hinzufügen und alles mit einem Mixer zu einem Teig verarbeiten. Anschließend zu einer Kugel formen und den Teig in Frischhaltefolie gewickelt mindestens 30 Minuten kaltstellen. Danach den Boden der Springform fetten. Die Hälfte des Teiges mit einem Nudelholz ausrollen und eine 26er Springform damit auskleiden. Dabei sollte ein ca. 2 cm hoher Rand entstehen. }
	
	\schritt{Für die Füllung}
	\zutat{250 g Butter }{in einem Topf zerlassen und abkühlen lassen. }
	\zutat{500 g 	Magerquark}{und}
	\zutat{200 g 	Zucker }{und}
	\zutat{1 Pck. 	Vanillezucker }{und}
	\zutat{3 	Eier}{und}
	\zutat{1 Pck. 	Vanillepuddingpulver }{und}
	\zutat{1 	Vanilleschote, das Mark davon}{dazu und mit einem Schneebesen zu einer gebundenen Masse verrühren, in die Form geben und glatt streichen. Den restlichen Teig in kleine Stücke zupfen und auf der Füllung verteilen. }
	
	\schritt{Backen}
	\zutat{Ober-/Unterhitze}{bei 180 \degree{}C}
	\zutat{Backzeit}{ca. 60 im unteren Drittel des Backofens. Nach dem Backen den Kuchen im geöffneten Backofen erkalten lassen.}
	
\end{rezept}


\end{document}